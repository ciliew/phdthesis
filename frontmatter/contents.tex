%*******************************************************
% Table of Contents
%*******************************************************
\refstepcounter{dummy}
\pdfbookmark[1]{\contentsname}{tableofcontents}
\thispagestyle{simple}
\tableofcontents*
%\clearpage
\cleardoublepage


%*******************************************************
% List of Figures, Tables, Listings, Acronyms, Notation
%*******************************************************

\begingroup 

%*******************************************************
% List of Figures
%*******************************************************  
%\refstepcounter{dummy}
%\pdfbookmark[1]{\listfigurename}{lof}
%\newlistof{listoffigures}{lof}{\listfigurename}
%\listoffigures*
%
%\clearpage
%\cleardoublepage

%*******************************************************
% List of Tables
%*******************************************************
%\refstepcounter{dummy}
%\pdfbookmark[1]{\listtablename}{lot}
%\newlistof{listoftables}{lot}{\listtablename}
%\listoftables*

%\vspace*{2.5cm}
%*******************************************************
% List of Listings
%*******************************************************      
%\refstepcounter{dummy}
%\pdfbookmark[1]{\listlistingsname}{lol}
%\newlistof{listoflistings}{lol}{\listlistingsname}
%\listoflistings*
%
%%\clearpage

%\cleardoublepage

%*******************************************************
% Acronyms
%*******************************************************
\iffalse
\refstepcounter{dummy}
\pdfbookmark[1]{Acronyms}{acronyms}
\chapter*{Acronyms}
\dado{sort alphabetically, update if needed or simply delete}
\begin{acronym}[NGCUCX]
	\acro{GP}{Gaussian Process}
	\acro{GPLVM}{Gaussian Process Latent Variable Model}
	\acro{PCA}{Principal Component Analysis}
	\acro{PPCA}{Probabilistic Principal Component Analysis}	
	\acro{kPCA}{Kernel Principal Component Analysis}
	\acro{SNE}{Stochastic Neighbourhood Embedding}
	\acro{t-SNE}{t-Distributed Stochastic Neighbourhood Embedding}
	\acro{MF}{Magnification Factor}
	\acro{Adam}{Adaptive Moment Estimation}
	\acro{VAE}{Variational Autoencoder}
	\acro{NN}{Neural Network}
	\acro{SAS}{Stochastic Active Sets}
	\acro{POLA}{Principle of Least Action}
\end{acronym}   
\fi
\iffalse
%*******************************************************
% Notation
%*******************************************************  
\refstepcounter{dummy}
\pdfbookmark[1]{Notation}{notation}
\chapter*{Notation}
\todo[inline]{Create table of notation for both Gp stuff and geometry stuff. This way I can make sure things are aligned.}
\section*{General}
$N$ is the number of observations \\
$D$ is the dimensionality of one observation/of data space/of the observed space \\
$q$ is the dimensionality of the latent space \\
$Y$ is data, this lives in the observed space so $Y \in \mathbb{R}^{N\times D}$, alternatively for one point we have $y_i \in \mathbb{R}^{D}$ \\
$X$ is a latent representation of data, this lives in the latent space so $X \in \mathbb{R}^{N\times D}$, alternatively for one point we have $y_i \in \mathbb{R}^{D}$ \\
We use $f$ to denote a map from the latent space to the bserved space such that $f: \mathbb{R}^q \to \mathbb{R}^D$ with $q < D$\\
$k(x,x')$ denotes the kernel that provides the covariance function. \\

I will denote derivatives in a number of ways. \marginnote{As we say in Danish, "dear child has many names}



\section*{GP}


\section*{Geometry}
Manifold \\
Tangent space \\
Curve \\
Christffel symbol \\
Uses Einstein notation \\

%\clearpage
\cleardoublepage
\fi


\endgroup